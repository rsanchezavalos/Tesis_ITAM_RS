\section{Tables}
\label{sec:pastwork:tables}

Tables are also quite important. Any table that can fit entirely on one page can be a floating table. If a table is longer and will span multiple pages, a long table can be inserted in-line with the text. This is demonstrated in Table~\ref{tab:usage:options}, and explained in Appendix~\ref{ch:implementation}.

Tables that fit on one page use normal floating figures. Keep the 'p' placement option (in addition to 'h' and 't') so that if the float cannot fit in-line with the document text, it can be on a separate page by itself immediately after it is placed. Without the 'p' option, the float may get pushed to the end of the chapter, along will all other floats in the chapter that follow it.

Table~\ref{table:pastwork:publishing} lists the various options for publishing your dissertation, with costs, as of 2010. You will have to bring a check for the appropriate amount, made out to ``Princeton University Library'', when you submit your bound dissertation copies to Mudd Library, along with the appropriate forms and the electronic copy of your dissertation burned to a CD (not a DVD) as a single PDF file. (See~\cite{muddthesis2009}.)

Traditional publishing is cheaper initially and lets you earn royalties if the publisher sells many copies of your dissertation. However, most of us won't have a best-seller dissertation and most likely won't earn royalties anyway. Instead, by choosing open access publishing, your dissertation will be available online for free to anyone who is interested. I strongly advocate for open access, to maximize the impact of your research.

Your dissertation is protected by copyright regardless of whether or not you have the copyright registered. However, registration establishes a public record of your copyright claim~\cite{muddthesis2009}. ProQuest will submit the copyright registration for an extra fee (about \$55). Alternatively, you can register it yourself at the Copyright Office's website for only \$35: \url{http://www.copyright.gov/eco/}.


\begin{table}[htbp]
\centering
\caption[Thesis Publishing Options]{Thesis publishing options~\cite{mudd2009}, as of May 2010. }
\label{table:pastwork:publishing}
\begin{tabular}{p{0.3\textwidth} p{0.15\textwidth} p{0.15\textwidth} p{0.15\textwidth} p{0.15\textwidth}}
\toprule
\textbf{Publishing Method} & \textbf{Publishing Fee}
 & \textbf{Diploma Fee} & \textbf{Copyright Registration Fee} & \textbf{Total} \\
\midrule
\multicolumn{5}{c}{Traditional Publishing}\\
\midrule

Traditional without copyright registration
& 65 & 15 & -- & 80 \\[0.2em]

Traditional with copyright registration
& 65 & 15 & 55 & 135 \\[0.2em]

\midrule
\multicolumn{5}{c}{Open Access}\\
\midrule

Open access without copyright registration
& 160 & 15 & -- & 175 \\[0.2em]

Open access with copyright registration
& 160 & 15 & 55 & 230 \\

\bottomrule
\end{tabular}
\end{table}